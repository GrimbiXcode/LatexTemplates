% das Papierformat zuerst
\documentclass[a4paper, 11pt]{article}
\usepackage[left=35mm,right=25mm,top=15mm,bottom=10mm,includeheadfoot]{geometry}

% hier beginnt das Dokument
\begin{document}
	% hier beginnt die Titelseite
	\begin{titlepage}

		\begin{center}
			% Oberer Teil der Titelseite:
			%\includegraphics[width=0.15\textwidth]{./logo}\\[1cm]    
			\textsc{\LARGE University of GrimbiXcode}\\[1.5cm]
			\textsc{\Large Final year project}\\[0.5cm]

			% Title
			\newcommand{\HRule}{\rule{\linewidth}{0.5mm}}
			\HRule \\[0.4cm]
			{ \huge \bfseries The reason of our Universe}\\[0.4cm]
			\HRule \\[1.5cm]
			% Author and supervisor
			\begin{minipage}{0.4\textwidth}
				\begin{flushleft} \large
					\emph{Author:}\\
					David \textsc{Grimbichler}
				\end{flushleft}
			\end{minipage}
			\hfill
			\begin{minipage}{0.4\textwidth}
				\begin{flushright} \large
					\emph{Supervisor:} \\
					Dr.~Dojo \textsc{Code}
				\end{flushright}
			\end{minipage}
			\vfill
			% Unterer Teil der Seite
			{\large \today}

		\end{center}
	\end{titlepage}

% Inhaltsverzeichnis neu Benennen
\renewcommand{\contentsname}{Inhalt}
% Inhaltsverzeichnis anzeigen
\tableofcontents

% Kapitel soll auf naechster Seite beginnen
\newpage

% Kapitelueberschrift
\section{Einleitung}

% Ueberschrift eines Abschnittes
\subsection{Motivation}

Dies ist ein Satz.

% Ueberschrift eines Abschnittes
\subsection{Abgrenzung}

Dies ist ein Satz.

% Kapitel soll auf naechster Seite beginnen
\newpage

% Kapitelueberschrift
\section{Theoretische Betrachtungen}

Dies ist ein Satz.

% Ende des Dokumentes
\end{document}