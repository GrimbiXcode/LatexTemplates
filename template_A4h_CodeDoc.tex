% das Papierformat zuerst
\documentclass[a4paper, 11pt]{article}

% ==========================================================================================
% Packages
% ==========================================================================================

% Für den gebarauch von Umlaute: Zeichenkodierung einstellen in
% Settings unten Links: windows-1252 statt UTF-8
% Einstellung für Umlaute
\usepackage[latin1]{inputenc} 

% Einstellungen des verwendeten Platzes auf dem Blatt
\usepackage[left=35mm,right=25mm,top=15mm,bottom=10mm,includeheadfoot]{geometry}

% ein (=1) und ausschalten (=0) von Bildern
\usepackage{ifthen}
\newboolean{enGraphics} %Deklaration
\setboolean{enGraphics}{true}%Zuweisung
\usepackage{graphicx}

% Damit das Einrücken verhindert wird
\parindent=0pt

% ==========================================================================================
% Snippet for Code Highlighting und anderes Zeugs
% ==========================================================================================
\usepackage{listings}
\usepackage{color}

\definecolor{dkgreen}{RGB}{51, 153, 51}
\definecolor{gray}{rgb}{0.4,0.4,0.4}
\definecolor{redstr}{RGB}{153, 0, 0}
\definecolor{black}{RGB}{0, 0, 0}

\lstdefinestyle{sharpc}{
	frame=single,
	language=[Sharp]C,
	aboveskip=3mm,
	belowskip=3mm,
	showstringspaces=false,
	columns=flexible,
	basicstyle={\small\ttfamily},
	numbers=none,
	numberstyle=\tiny\color{gray},
	keywordstyle=\color{blue},
	commentstyle=\color{dkgreen},
	stringstyle=\color{redstr},
	breaklines=true,
	breakatwhitespace=true,
	aboveskip=5pt,
	belowskip=5pt,
	tabsize=3,
	rulecolor=\color{black}
}

\lstdefinestyle{synless}{
	language={},                % {} für normalen Klartext
	backgroundcolor=\color{white},
	linewidth=\linewidth,       % Zeilenbreite
	columns=flexible,
	basicstyle={\small\ttfamily\color{gray}},
	aboveskip=5pt,
	belowskip=5pt,
	breaklines=true,             % Zeileumbruch
	breakatwhitespace=true  %Umbruch an Leerzeichen
}

\lstdefinestyle{important}{
	language={},                % {} für normalen Klartext
	frame=single,
	backgroundcolor=\color{white},
	linewidth=\linewidth,       % Zeilenbreite
	columns=flexible,
	basicstyle=\bfseries,
	aboveskip=5pt,
	belowskip=5pt,
	breaklines=true,             % Zeileumbruch
	breakatwhitespace=true  %Umbruch an Leerzeichen
}

% ==========================================================================================
% Titelblatt
% ==========================================================================================
% hier beginnt das Dokument
\begin{document}
	% hier beginnt die Titelseite
	\begin{titlepage}

		\begin{center}
			% Oberer Teil der Titelseite:
			
			% place here your logo or title image
			%\includegraphics[width=0.15\textwidth]{./logo}\\[1cm]    
			\textsc{\LARGE University of GrimbiXcode}\\[1.5cm]
			\textsc{\Large Final year project}\\[0.5cm]

			% Title
			\newcommand{\HRule}{\rule{\linewidth}{0.5mm}}
			\HRule \\[0.4cm]
			{ \huge \bfseries The reason of our Universe}\\[0.4cm]
			\HRule \\[1.5cm]
			% Author and supervisor
			\begin{minipage}{0.4\textwidth}
				\begin{flushleft} \large
					\emph{Author:}\\
					David \textsc{Grimbichler}
				\end{flushleft}
			\end{minipage}
			\hfill
			\begin{minipage}{0.4\textwidth}
				\begin{flushright} \large
					\emph{Supervisor:} \\
					Dr.~Dojo \textsc{Code}
				\end{flushright}
			\end{minipage}
			\vfill
			% Unterer Teil der Seite
			{\large \today}

		\end{center}
	\end{titlepage}

% ==========================================================================================
% Inhaltsverzeichnis / table of content
% ==========================================================================================
% Benenne das Inahltsverzeichniss neu
\renewcommand{\contentsname}{Inhalt}
% Inhaltsverzeichnis anzeigen
\tableofcontents

% Kapitel soll auf naechster Seite beginnen
% ******************************************************************************************
% New Page
\newpage
% ******************************************************************************************

% ==========================================================================================
% Eigentlicher Dokumentinhalt / Document content
% ==========================================================================================
% Kapitelueberschrift
\section{Einleitung}

% Ueberschrift eines Abschnittes
\subsection{Motivation}

Dies ist ein Satz.

% Ueberschrift eines Abschnittes
\subsection{Abgrenzung}

Dies ist ein Satz.

Neuer Code mit gruener Einfaerbung:

\begin{lstlisting}[style=sharpc]
// MainWindow.xaml.cs
_tboErgebnisBottom.Foreground = Brushes.Green;
_tboErgebnisTop.Foreground = Brushes.Green;
_tboErgebnisTop.Enabled = true;
_tboErgebnisTop.Text = "Erfolgreich";
\end{lstlisting}

% Ueberschrift eines Abschnittes
\subsection{Allgemeine Ablage der Testsoftware} \label{sssec:num1}
Die Testsoftware ist bis zur Version 0.9 im Verzeichnis:

\begin{lstlisting}[style=synless]
X:\Produkte\Entwicklung\Software\Testsoftware\
\end{lstlisting}


% Kapitel soll auf naechster Seite beginnen
% ******************************************************************************************
% New Page
\newpage
% ******************************************************************************************
% Kapitelueberschrift
\section{Theoretische Betrachtungen}

Dies ist ein Satz.

% ==========================================================================================
% Ende des Dokuments/ End of document
% ==========================================================================================
\end{document}